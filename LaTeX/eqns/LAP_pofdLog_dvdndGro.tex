\begin{equation*}\begin{gathered}\begin{aligned}

The same derivations as before lead to the conclusion that

This problem is not solved by relaxing the assumption that dividends are IID; when dividends are positively serially correlated it remains the case that when dividends are above their average level, an increase in risk aversion causes an increase in asset prices.

This is likely one of the reasons that more exotic specifications of preferences have gained popularity in recent years; the two most prominent examples are Epstein-Zin utility, and various kinds of habit formation.

\input handoutBibMake

\end{document}


\begin{comment}
Defining $\exexp[\alpha \bullet] = \Ex[(e^{\bullet})^\alpha]$ as the expectation of the random variable $e^{\alpha \bullet}$, mathfact \ELogNorm~ says that
\begin{equation}\begin{gathered}\begin{aligned}
  \exexp[\alpha \err]  & =  e^{\alpha \mu + (1/2) \alpha^{2} \sigma^{2}}
\end{aligned}\end{gathered}\end{equation}
\end{comment}


\begin{equation}\begin{gathered}\begin{aligned}
  \dvdnd_{t}^{\CRRA}\edvdnd & =  \dvdnd_{t}^{\CRRA}\bar{\dvdnd}^{-\CRRA}\bar{\dvdnd}(e^{-(1/2)\CRRA \sigma^{2}})^{1-\CRRA} \edvdnd
\\ & =  \left(\frac{\dvdnd_{t}}{\bar{\dvdnd}}\right)^{\CRRA}\bar{\dvdnd}(e^{-(1/2)\CRRA \sigma^{2}})^{1-\CRRA} \edvdnd
\end{aligned}\end{gathered}\end{equation}

This equation yields a few insights.

First, since everything in the equation is a constant except $\dvdnd_{t}$ it can be rewritten as $\Price_{t} = \dvdnd_{t}^{\CRRA} \omega$ for some constant $\omega$.  Since $\CRRA>1$, this means that prices move more than one-for-one with dividends.


Noting that everything in this expression is a constant except $\dvdnd_{t}$, the expression yields a few insights:
\begin{enumerate}
\item Prices are more volatile than dividends (because $\CRRA > 1$)
\end{enumerate}


